\documentclass[12pt, dvipdfmx]{article}
\usepackage[margin=1in]{geometry}
\usepackage{enumitem}
\usepackage{bm}

\title{
  \vspace{-2cm}
  CS 224n Assignment \#3 \\
  \author{Yoshihiro Kumazawa}
}

\begin{document}
\maketitle
\begin{enumerate}[label=\textbf{\arabic*.}]
\item \textbf{Machine Learning \& Neural Networks}
\begin{enumerate}[label=(\alph*)]
\item
\begin{enumerate}[label=\roman*.]
\item When $\beta_1$ is large, $\bm{m}$ relies more on the history of the past gradients rather than the new one. For example, if $\beta_1=0.9$, the contribution of the new gradient to weight update is only 10\% of that without momentum.
\item
\end{enumerate}
\item
\begin{enumerate}[label=\roman*.]
\item
\item
\end{enumerate}
\end{enumerate}
\item \textbf{Neural Transition-Based Dependency Parsing}
\begin{enumerate}[label=(\alph*)]
\item
\item
\item
\item
\item
\item
\end{enumerate}
\end{enumerate}
\end{document}
