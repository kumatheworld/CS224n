\documentclass[12pt]{article}
\usepackage[margin=1in]{geometry}
\usepackage{enumitem}
\usepackage{bm}
\usepackage{amsmath}
\usepackage{amsfonts}

\title{
  \vspace{-2cm}
  CS 224n Assignment \#5 \\
  \author{Yoshihiro Kumazawa}
}

\begin{document}
\maketitle
\begin{enumerate}[label=\textbf{\arabic*.}]
  \item \textbf{Character-based convolutional encoder for NMT}
  \begin{enumerate}[label=(\alph*)]
    \item Convolutional architectures can operate over variable length input too since convolutional layers slide fixed-sized windows over input unlike linear layers.
    \item The size of the padding should be 1 so that the padded vector will have size at least 5. Indeed, $m_\text{word}$ could be 1 if all words in a batch happen to be some characters of length 1 like 'a', in which case we have $\mathbf{x}'_\text{padded}\in\mathbb{Z}^3$.
    \item The highway layer makes it possible to combine local features and global features. In other words, it matches our intuition that we can sometimes understand the meaning of a word by just looking at a little chunk of consecutive characters at a time but it is sometimes better to consider the whole characters in it at once. In order to simplify the network semantics in the beginning of training, I would initialize $\bm{b}_\text{gate}$ to be negative.
    \item Transformers are easier to parallelize and faster to train.
    \item See \texttt{vocab.py}.
    \item For the highway network implementation, see \texttt{highway.py}. I added a function \texttt{question\_1f\_sanity\_check()} in \texttt{sanity\_check.py} to test the following expected properties.
    \begin{itemize}
      \item The output size is correct for a given input.
      \item $\bm{x}_\text{highway}=\bm{x}_\text{conv\_out}$ when $\bm{x}_\text{gate}=0$, which is checked by making $\bm{b}_\text{gate}=-\infty$.
      \item $\bm{x}_\text{highway}=\bm{x}_\text{proj}$ when $\bm{x}_\text{gate}=1$, which is checked by making $\bm{b}_\text{gate}=\infty$.
      \item $\bm{x}_\text{highway}=\bm{x}_\text{conv\_out}$ when the projection layer is the identity function.
    \end{itemize}
    In addition, I checked if $\bm{x}_\text{gate}$ is initialized to be negative by computing the mean 4 times. (see my answer for 1 (c) above).
    \item For the convolutional network implementation, see \texttt{cnn.py}. I added a function \texttt{question\_1g\_sanity\_check()} in \texttt{sanity\_check.py} to test the following expected properties.
    \begin{itemize}
      \item The sizes of input channels, output channels, kernels and padding of the convolutional layer are correct.
      \item The output size is correct for a given input.
    \end{itemize}
    \item
    \item
    \item
  \end{enumerate}
  \item \textbf{Character-based LSTM decoder for NMT}
  \begin{enumerate}[label=(\alph*)]
    \item
    \item
    \item
    \item
    \item
  \end{enumerate}
  \item \textbf{Analyzing NMT Systems}
  \begin{enumerate}[label=(\alph*)]
    \item
    \item
    \begin{enumerate}[label=\roman*.]
      \item
      \item
      \item
    \end{enumerate}
    \item
  \end{enumerate}
\end{enumerate}
\end{document}
